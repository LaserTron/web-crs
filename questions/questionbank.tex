%%%%%%%%%%%%%%%%%%%%%%%%%%%%%%%%%%%%%%%%%%%%%%%%%%%%%%%%%%%%%%%%%%%%%%%%%%
% This is the bank of clicker questions. They are tagged
% and organized by topic. For now each question will have:
% 1)A commented identifier consisting of the author's initial and a
% number
% 2) A list of notions
% 3) A question statement
% 4) Answer choices (enclosed in an enumerate environment)
% 5) The correct answer
% 
% N.b. pdf, jpeg, png images are all acceptable
%
% The goal is to have a list that easy to chose from.
% Any figures should be named w.r.t. the identifier of the
% corresponding question. WHEN ADDING YOUR OWN QUESTIONS PLEASE
% USE THE APPROPRIATE FORMATTING CONVENTIONS.
%%%%%%%%%%%%%%%%%%%%%%%%%%%%%%%%%%%%%%%%%%%%%%%%%%%%%%%%%%%%%%%%%%%%%%%%%%



\documentclass{article}
\usepackage{geometry,amssymb,graphicx,enumerate}
%\usepackage[all]{xy}

\newenvironment{clkrQuestion}[2]{
  \begin{minipage}{\textwidth}
    \vskip .4in
  \textbf{Id:}\texttt{#1}\\ 
  \textbf{Tags:}{~#2}\\
}
{\end{minipage}}

%These funny commands  are meant to make it easier to convert questions to other formats
\newcommand{\answer}[1]{\textbf{Answer: }\ref{#1}}
\newcommand{\explanation}[1]{#1}
\newcommand{\probstmt}[1]{#1}

\title{clicker question bank}
\begin{document}
\maketitle

\begin{clkrQuestion}{NT0}{asymptotes, graphing, continuity}
  Let $f(x)$ be continuous on $(0,\infty)$. True or false. If \[
  \lim_{x\to 0^+}f(x)=\infty, \lim_{x\to \infty}f(x)=-3
  \] then $f(x)$ must be decreasing on $(0,\infty).$
  \begin{enumerate}[A.]
  \item\label{NT0A} True
  \item\label{NT0B} False
  \end{enumerate}
  %
  \answer{NT0B}
  \explanation{Draw a graph.}
\end{clkrQuestion}

\begin{clkrQuestion}{NT1}{asymptotes, graphing, continuity}
  Let $f(x)$ be continuous on $(0,\infty)$. True or false. If \[
  \lim_{x\to 0^+}f(x)=\infty, \lim_{x\to \infty}f(x)=-3
  \] and if $f(x)$ is decreasing on $(0,\infty)$ , then for any $c>0$
  we never have $f(c)=-4$.
  \begin{enumerate}[A.]
  \item\label{NT1A} True
  \item\label{NT1B} False
  \end{enumerate}
  %
  \answer{NT1A}
  \explanation{Draw a graph.}
\end{clkrQuestion}

\begin{clkrQuestion}{NT2}{asymptotes, graphing, continuity}
  Let $f(x)$ be continuous on $(0,\infty)$. True or false. If \[
  \lim_{x\to 0^+}f(x)=\infty, \lim_{x\to \infty}f(x)=-3
  \] then there must be some $c>0$ such that $f(c)=0$.
  \begin{enumerate}[A.]
  \item\label{NT2A} True
  \item\label{NT2B} False
  \end{enumerate}
  %
  \answer{NT2A}
  \explanation{Draw a graph.}
\end{clkrQuestion}

\begin{clkrQuestion}{NT3}{asymptotes, graphing, continuity}
  Let $f(x)$ be continuous on $(0,\infty)$. True or false. If \[
  \lim_{x\to 0^+}f(x)=-\infty, \lim_{x\to \infty}f(x)=3
  \] then it is impossible for there to be some $c>0$ such that $f(c)=4$.
  \begin{enumerate}[A.]
  \item\label{NT3A} True
  \item\label{NT3B} False
  \end{enumerate}
  %
  \answer{NT3B}
  \explanation{The graph may bump up above 4.}
\end{clkrQuestion}

\begin{clkrQuestion}{NT4}{asymptotes, graphing, continuity}
  Let $f(x)$ be defined on $(0,\infty)$. True or false. If
  \[ \lim_{x\to 0^+}f(x)=-\infty, \lim_{x\to \infty}f(x)=3
  \] then there must be some $c>0$ such that $f(c)=0$.
  \begin{enumerate}[A.]
  \item\label{NT4A} True
  \item\label{NT4B} False
  \end{enumerate}
  %
  \answer{NT4B}
  \explanation{The graph may have a jump discontinuity.}
\end{clkrQuestion}

\begin{clkrQuestion}{NT5}{instantaneous velocity, units}
  The position $p(t)$ of a particle (in meters) as a function of time
  $t$ (in seconds) is given by the following table:
  \[\left|
    \begin{array}{c|c|c|c|c|c|c|c|c|c}
      \hline
      t&1&2&3&4&5&6&7&8&9\\
      \hline
      p(t)&1&2&3&4&5&6&7&8&9\\      
      \hline
    \end{array}
  \right|.\]
  True or false. For $1\leq t \leq 9$, the particle
  must have been  travelling at $3.6$ km/h.
  \begin{enumerate}[A.]
  \item\label{NT5A} True 
  \item\label{NT5B} False 
  \end{enumerate}
  \answer{NT5B}
\end{clkrQuestion}

\begin{clkrQuestion}{NT6}{instantaneous velocity, units}
  The position $p(t)$ of a particle (in meters) as a function of time
  $t$ (in seconds) is given by the following table:
  \[\left|
    \begin{array}{c|c|c|c|c|c|c|c|c|c}
      \hline
      t&1&2&3&4&5&6&7&8&9\\
      \hline
      p(t)&1&2&3&4&5&6&7&8&9\\      
      \hline
    \end{array}
  \right|.\]
  True or false. There is some $1\leq c \leq 9$ such that at $t=c$ the
  particle was travelling at least at $36$ km/h. (Teleportation is not
  allowed.)
  \begin{enumerate}[A.]
  \item\label{NT6A} True 
  \item\label{NT6B} False 
  \end{enumerate}
  \answer{NT6A}
\end{clkrQuestion}

\begin{clkrQuestion}{NT7}{continuity}
    Which of the following functions are not continuous?
    \begin{enumerate}
        \item\label{NT7A} Gravitational force as a function of distance from a star.
        \item\label{NT7B} Population of a city as a function of time.
        \item\label{NT7C} The mass of a beaker being filled with water as a function of the water level.
        \item\label{NT7D} The absolute value function
    \end{enumerate}
    \answer{NT7B}
\end{clkrQuestion}

\begin{clkrQuestion}{NT8}{average velocity, exponents}
    Using the formula $KE=\frac{1}{2}m v^2$ and the kinetic theory of  gases, we have the following formula for a nitrogen molecule at room pressure and temperature:
    \[
    v = \sqrt{\frac{2KE}{m}}    
    \] where the mass of a molecule is $m=4.65 \cdot 10^{-26}$ kg and we have the kinetic energy $2KE = 1.25 \cdot 10^{-20}$ J. Of what order of magnitude is the average velocity of a nitrogen molecule in this room over one second?
    \begin{enumerate}[A.]
        \item\label{NT8A} meters per second.
        \item\label{NT8B} 100s of meters per second.
        \item\label{NT8C} 10,000s of meters per second.
        \item\label{NT8D} 100,000s of meters per second.
    \end{enumerate}
    \answer{NT8A}
    \explanation{The wind speed in the room is of the order of one meter per second. If we look at the average velocity over very small times, however, we will approach something to the order 100 m/s.}
\end{clkrQuestion}

\begin{clkrQuestion}{NT9}{quotient rule, pitfalls}
    The following is correct\[
    \frac d {dx} \left(\frac{x^2}{\pi} \right)= \frac{2x\pi -x^2\cdot 0}{\pi^2}
    \]  
    \begin{enumerate}[A.]
    \item\label{NT9A} True
    \item\label{NT9B} False
    \end{enumerate}
    \answer{NT9A}
    \explanation{It is correct, but dumb. $\pi$ is a constant. }
\end{clkrQuestion}

\begin{clkrQuestion}{NT10}{quotient rule, pitfalls}
    The following is correct\[
    \frac d {dx} \left(\frac{\sqrt{x}}{x} \right)= \frac{\sqrt{x} - \left(\frac 1 2 \cdot \frac{1}{\sqrt x}\cdot x\right)}{x}
    \]  
    \begin{enumerate}[A.]
    \item\label{NT10A} True
    \item\label{NT10B} False
    \end{enumerate}
    \answer{NT10B}
    \explanation{There is a sign mistake. Also this is a dumb way to do it. This expression can be simplified first.}
\end{clkrQuestion}

\begin{clkrQuestion}{NT11}{Extreme Value Theorem}
    Let $f(x)$ be a function that is increasing on the interval $[a,b]$ which of the following statements is correct?
    \begin{enumerate}
        \item\label{NT11A} If $f$ is differentiable and  $f'(x) >0$ on $[a,b]$, then no global maximum or minimum is attained.
        \item\label{NT11B} $f$ attains a global minimum at $b$.
        \item\label{NT11C} $f$ may attain global maximum inside $(a,b)$.
        \item\label{NT11D} Nothing can be said, since $f$ is not given.
    \end{enumerate}
    \answer{NT11B}
\end{clkrQuestion}

\end{document}